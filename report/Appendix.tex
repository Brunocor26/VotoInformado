\chapter{Anexos} \label{chap:appendix}

Esta secção contém excertos de código relevantes que detalham a implementação do sistema VotoInformado.

\section{Código Fonte Relevante}

\subsection*{Cliente da API - ApiClient.java}
\begin{lstlisting}[language=Java, caption=Cliente da API para comunicação com o servidor.]
package pt.ubi.pdm.votoinformado.api;

import android.util.Log;
import java.io.IOException;
import java.net.HttpURLConnection;
import java.net.URL;
import java.util.concurrent.CountDownLatch;
import retrofit2.Retrofit;
import retrofit2.converter.gson.GsonConverterFactory;

public class ApiClient {
    private static final String RENDER_URL = "https://api-votoinformado.onrender.com/";
    private static final String LOCAL_URL = "http://10.0.2.2:3000/"; // Localhost for Android Emulator
    private static final String LOCAL_IP_URL = "http://10.250.134.7:3000/"; // Local IP for Physical Device

    private static volatile ApiClient instance;
    private static final Object lock = new Object();
    private final Retrofit retrofit;
    private static String baseUrl;

    private ApiClient(Retrofit retrofit) {
        this.retrofit = retrofit;
    }

    private static boolean isUrlAvailable(String urlString) {
        try {
            URL url = new URL(urlString);
            HttpURLConnection connection = (HttpURLConnection) url.openConnection();
            connection.setRequestMethod("HEAD");
            connection.setConnectTimeout(15000); // 15 seconds timeout for Render cold start
            connection.setReadTimeout(15000);
            connection.connect();
            int responseCode = connection.getResponseCode();
            return (responseCode >= 200 && responseCode < 300);
        } catch (IOException e) {
            return false;
        }
    }

    public static ApiClient getInstance() {
        if (instance == null) {
            synchronized (lock) {
                if (instance == null) {
                    CountDownLatch latch = new CountDownLatch(1);
                    new Thread(() -> {
                        if (isUrlAvailable(RENDER_URL)) {
                            baseUrl = RENDER_URL;
                        } else if (isUrlAvailable(LOCAL_IP_URL)) {
                            baseUrl = LOCAL_IP_URL;
                        } else {
                            baseUrl = LOCAL_URL;
                        }
                        Log.d("ApiClient", "Using base URL: " + baseUrl);

                        Retrofit retrofit = new Retrofit.Builder()
                                .baseUrl(baseUrl)
                                .addConverterFactory(GsonConverterFactory.create())
                                .build();
                        
                        instance = new ApiClient(retrofit);
                        latch.countDown();
                    }).start();

                    try {
                        latch.await(); // Block until initialization is complete
                    } catch (InterruptedException e) {
                        Thread.currentThread().interrupt();
                        // Handle error, maybe fallback to a default
                        if (instance == null) {
                           Log.e("ApiClient", "Initialization interrupted, falling back to local URL");
                           baseUrl = LOCAL_URL;
                           Retrofit retrofit = new Retrofit.Builder()
                                   .baseUrl(baseUrl)
                                   .addConverterFactory(GsonConverterFactory.create())
                                   .build();
                           instance = new ApiClient(retrofit);
                        }
                    }
                }
            }
        }
        return instance;
    }

    public ApiService getApiService() {
        return retrofit.create(ApiService.class);
    }

    public static String getBaseUrl() {
        if (instance == null) {
            getInstance(); // ensure initialization
        }
        return baseUrl;
    }
}
\end{lstlisting}

\subsection*{Atividade Principal - HomeActivity.java}
\begin{lstlisting}[language=Java, caption=Atividade principal que gere a navegação e o fragmento inicial.]
package pt.ubi.pdm.votoinformado.activities;

import android.Manifest;
import android.content.Intent;
import android.content.pm.PackageManager;
import android.os.Build;
import android.os.Bundle;
import android.view.MenuItem;

import androidx.activity.result.ActivityResultLauncher;
import androidx.activity.result.contract.ActivityResultContracts;
import androidx.annotation.NonNull;
import androidx.appcompat.app.AppCompatActivity;
import androidx.core.content.ContextCompat;
import androidx.fragment.app.Fragment;
import androidx.work.ExistingPeriodicWorkPolicy;
import androidx.work.PeriodicWorkRequest;
import androidx.work.WorkManager;

import com.google.android.material.bottomnavigation.BottomNavigationView;

import java.util.concurrent.TimeUnit;

import pt.ubi.pdm.votoinformado.R;
import pt.ubi.pdm.votoinformado.activities.notificacoes.SyncDatesWorker;
import pt.ubi.pdm.votoinformado.fragments.CandidatosFragment;
import pt.ubi.pdm.votoinformado.fragments.ChooseEventTypeFragment;
import pt.ubi.pdm.votoinformado.fragments.HomeFragment;
import pt.ubi.pdm.votoinformado.fragments.NoticiasFragment;
import pt.ubi.pdm.votoinformado.fragments.PeticoesFragment;
import pt.ubi.pdm.votoinformado.fragments.SondagensFragment;

public class HomeActivity extends AppCompatActivity implements BottomNavigationView.OnNavigationItemSelectedListener {

    // Launcher para pedir a permissão de notificação
    private final ActivityResultLauncher<String> requestPermissionLauncher = registerForActivityResult(new ActivityResultContracts.RequestPermission(), isGranted -> {
    });

    @Override
    protected void onCreate(Bundle savedInstanceState) {
        super.onCreate(savedInstanceState);

        // Check if user is logged in via shared preferences
        android.content.SharedPreferences prefs = getSharedPreferences("user_session", MODE_PRIVATE);
        String token = prefs.getString("auth_token", null);

        if (token == null || token.isEmpty()) {
            startActivity(new Intent(this, LoginActivity.class));
            finish();
            return;
        }

        setContentView(R.layout.activity_home);

        BottomNavigationView bottomNav = findViewById(R.id.bottom_navigation);
        bottomNav.setOnNavigationItemSelectedListener(this);

        // Load the default fragment
        if (savedInstanceState == null) {
            loadFragment(new HomeFragment());
        }

        // Pede permissão de notificações (para Android 13+)
        askNotificationPermission();
        
        //funcao que vai tratar das notificacoes mesmo sem a necessidade de ter a aplicacao aberta
        scheduleDateSync();
    }

    private void askNotificationPermission() {
        // Apenas necessário para API 33+ (Android 13)
        if (Build.VERSION.SDK_INT >= Build.VERSION_CODES.TIRAMISU) {
            if (ContextCompat.checkSelfPermission(this, Manifest.permission.POST_NOTIFICATIONS) != PackageManager.PERMISSION_GRANTED) {
                // Pede diretamente a permissão
                requestPermissionLauncher.launch(Manifest.permission.POST_NOTIFICATIONS);
            }
        }
    }

    private void scheduleDateSync() {
        //criamos um pedido de trabalho periódico
        PeriodicWorkRequest syncDatesRequest =
                new PeriodicWorkRequest.Builder(SyncDatesWorker.class, 24, TimeUnit.HOURS)
                        .build();               //executar a logica do SyncDatesWorker.class a cada 24h

        //entregamos o pedido de trabalho ao WorkManager
        WorkManager.getInstance(this).enqueueUniquePeriodicWork(
                "syncDatesWork",
                ExistingPeriodicWorkPolicy.KEEP,
                syncDatesRequest);
    }

    private boolean loadFragment(Fragment fragment) {
        if (fragment != null) {
            getSupportFragmentManager().beginTransaction()
                    .replace(R.id.fragment_container, fragment)
                    .commit();
            return true;
        }
        return false;
    }

    @Override
    public boolean onNavigationItemSelected(@NonNull MenuItem item) {
        Fragment fragment = null;

        int itemId = item.getItemId();
        if (itemId == R.id.nav_home) {
            fragment = new HomeFragment();
        } else if (itemId == R.id.nav_candidatos) {
            fragment = new CandidatosFragment();
        } else if (itemId == R.id.nav_eventos) {
            fragment = new pt.ubi.pdm.votoinformado.fragments.ImportantDatesHostFragment();
        } else if (itemId == R.id.nav_sondagens) {
            fragment = new SondagensFragment();
        } else if (itemId == R.id.nav_noticias) {
            fragment = new NoticiasFragment();
        } else if (itemId == R.id.nav_peticoes) {
            fragment = new PeticoesFragment();
        }

        return loadFragment(fragment);
    }
}
\end{lstlisting}

\subsection*{Modelo de Dados - Candidato.java}
\begin{lstlisting}[language=Java, caption=Classe que representa o modelo de dados de um Candidato.]
package pt.ubi.pdm.votoinformado.classes;

import android.annotation.SuppressLint;
import android.content.Context;
import android.util.Log;

    // @Exclude removed

import java.io.Serializable;

import pt.ubi.pdm.votoinformado.R;

import com.google.gson.annotations.SerializedName;

public class Candidato implements Serializable {

    @SerializedName("_id")
    private String id;

    @SerializedName("id")
    private String stringId;

    private String nome;
    private String partido;
    // private String fotoNome;
    private String profissao;
    private String cargosPrincipais;
    private String biografiaCurta;
    private String siteOficial;

    public Candidato() {
        // Construtor vazio necessário para o Firestore
    }

    // Getters and Setters with PropertyName annotations

    public String getId() { return id; }
    public void setId(String id) { this.id = id; }

    public String getStringId() { return stringId; }
    public void setStringId(String stringId) { this.stringId = stringId; }

    public String getNome() { return nome; }
    public void setNome(String nome) { this.nome = nome; }

    public String getPartido() { return partido; }
    public void setPartido(String partido) { this.partido = partido; }

    // fotoNome removed


    public String getProfissao() { return profissao; }
    public void setProfissao(String profissao) { this.profissao = profissao; }

    public String getCargosPrincipais() { return cargosPrincipais; }
    public void setCargosPrincipais(String cargosPrincipais) { this.cargosPrincipais = cargosPrincipais; }

    public String getBiografiaCurta() { return biografiaCurta; }
    public void setBiografiaCurta(String biografiaCurta) { this.biografiaCurta = biografiaCurta; }

    public String getSiteOficial() { return siteOficial; }
    public void setSiteOficial(String siteOficial) { this.siteOficial = siteOficial; }

    @SerializedName("photoUrl")
    private String photoUrl;

    public String getPhotoUrl() {
        return photoUrl;
    }

    public void setPhotoUrl(String photoUrl) {
        this.photoUrl = photoUrl;
    }
}
\end{lstlisting}