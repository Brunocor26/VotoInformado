\chapter{Arquitetura e Tecnologias} \label{chap:arquitetura}

Este capítulo descreve a arquitetura de software adotada para o desenvolvimento da aplicação VotoInformado, bem como as tecnologias e bibliotecas utilizadas.

\section{Arquitetura do Sistema} \label{sec:arq_sistema}
A aplicação foi desenvolvida para o sistema operativo Android \cite{AndroidDev} e utiliza a linguagem Java. A estrutura do projeto segue os padrões recomendados pela Google e organiza o código em componentes lógicos para facilitar a manutenção e escalabilidade.

A arquitetura baseia-se na separação de responsabilidades:
\begin{itemize}
    \item \textbf{Activities e Fragments}: Responsáveis pela interface de utilizador (UI) e interação com o utilizador. Exemplos incluem \texttt{HomeActivity}, \texttt{NoticiasFragment} e \texttt{CandidatosFragment}.
    \item \textbf{Adapters}: Fazem a ponte entre os dados e os componentes de visualização de listas (\texttt{RecyclerView}).
    \item \textbf{Modelos (Classes)}: Representam as entidades de dados, como \texttt{Candidato}, \texttt{Noticia} e \texttt{Sondagem}.
    \item \textbf{Utils/Helpers}: Classes utilitárias para acesso a dados e operações comuns.
\end{itemize}

\section{Tecnologias Utilizadas} \label{sec:arq_tecnologias}

\subsection{Backend (API Customizada)}
O backend da aplicação, inicialmente projetado em Firebase, foi migrado para uma solução personalizada para responder a requisitos de maior complexidade técnica e controlo. A arquitetura atual utiliza:
\begin{itemize}
    \item \textbf{Node.js \& Express} \cite{NodeJS, ExpressJS}: Servidor de aplicação que expõe uma API RESTful para comunicação com a aplicação móvel.
    \item \textbf{MongoDB} \cite{MongoDB}: Base de dados NoSQL orientada a documentos, escolhida pela sua flexibilidade e escalabilidade.
    \item \textbf{Mongoose} \cite{Mongoose}: Biblioteca de modelação de objetos (ODM) para interação estruturada com o MongoDB.
\end{itemize}

\subsection{Firebase (Legado/Autenticação)}
Embora a persistência de dados tenha sido migrada, o Firebase continua a desempenhar um papel auxiliar:
\begin{itemize}
    \item \textbf{Firebase Authentication}: Mantido para gestão segura de identidades e sessões de utilizador.
    \item \textbf{Nota de Migração}: A transição do Firebase Firestore para a API personalizada foi motivada por uma sugestão pedagógica para aumentar a complexidade do projeto e demonstrar competências de desenvolvimento \textit{full-stack}.
\end{itemize}

\subsection{Bibliotecas Externas}
Para enriquecer a funcionalidade da aplicação, foram integradas diversas bibliotecas \textit{open-source}:

% [TODO: Tabela de Bibliotecas]
\begin{table}[H]
    \centering
    \caption{Bibliotecas externas utilizadas no projeto.}
    \begin{tabular}{|l|l|}
        \hline
        \textbf{Biblioteca} & \textbf{Propósito} \\
        \hline
        Picasso & Carregamento e colocação em cache de imagens assíncronos. \\
        \hline
        MPAndroidChart & Criação de gráficos para visualização de sondagens. \\
        \hline
        CircleImageView & Exibição de imagens de perfil circulares. \\
        \hline
        Retrofit/Gson & (Legado) Análise sintática de dados JSON. \\
        \hline
    \end{tabular}
    \label{tab:bibliotecas}
\end{table}

\begin{itemize}
    \item \textbf{Picasso} \cite{Picasso}: Biblioteca poderosa para o carregamento e cache de imagens em Android. Resolve problemas complexos de gestão de memória e reciclagem de vistas em listas.
    \item \textbf{MPAndroidChart} \cite{MPAndroidChart}: Utilizada para a criação de gráficos interativos e visualmente apelativos na secção de sondagens.
    \item \textbf{CircleImageView}: Permite a exibição de imagens de perfil com formato circular de forma simples e eficiente.
    \item \textbf{Retrofit/Gson} \cite{Retrofit} (Legado): Inicialmente utilizadas para análise sintática de JSON, foram substituídas pela integração direta com o Firestore, mas fizeram parte do processo de desenvolvimento.
\end{itemize}