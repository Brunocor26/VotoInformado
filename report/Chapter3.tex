\chapter{Funcionalidades} \label{chap:funcionalidades}

A aplicação VotoInformado oferece um conjunto de funcionalidades desenhadas para informar o eleitor.

\section{Autenticação e Perfil} \label{sec:func_auth}
A segurança e personalização são garantidas através de um sistema de autenticação robusto.

% [TODO: Inserir Screenshot Login]
\begin{figure}[H]
    \centering
    \includegraphics[width=0.4\textwidth]{images/screenshot_login.png.jpg}
    \caption{Ecrã de Login da aplicação.}
    \label{fig:login}
\end{figure}

\begin{itemize}
    \item \textbf{Login e Registo}: Os utilizadores podem criar conta com email e password.
    \item \textbf{Gestão de Sessão}: A aplicação mantém a sessão do utilizador ativa, o que permite acesso direto sem necessidade de login constante.
\end{itemize}

\section{Requisitos Funcionais} \label{sec:func_requisitos}
A tabela abaixo resume as principais funcionalidades implementadas.

% [TODO: Tabela de Requisitos]
\begin{table}[H]
    \centering
    \caption{Requisitos Funcionais do VotoInformado.}
    \begin{tabular}{|c|l|p{8cm}|}
        \hline
        \textbf{ID} & \textbf{Funcionalidade} & \textbf{Descrição} \\
        \hline
        RF01 & Autenticação & Permitir login e registo via Email/Password. \\
        \hline
        RF02 & Listar Candidatos & Visualizar lista de candidatos com foto e partido. \\
        \hline
        RF03 & Detalhe Candidato & Ver biografia, propostas e cargos de um candidato. \\
        \hline
        RF04 & Feed Notícias & Ler notícias políticas atualizadas via RSS. \\
        \hline
        RF05 & Sondagens & Visualizar gráficos de intenção de voto. \\
        \hline
    \end{tabular}
    \label{tab:requisitos}
\end{table}


\section{Consulta de Candidatos} \label{sec:func_candidatos}
Esta é uma das funcionalidades centrais da aplicação.

% [TODO: Inserir Screenshot Lista Candidatos]
\begin{figure}[H]
    \centering
    \includegraphics[width=0.4\textwidth]{images/screenshot_candidatos.jpg}
    \caption{Lista de candidatos apresentada ao utilizador.}
    \label{fig:candidatos}
\end{figure}

\begin{itemize}
    \item \textbf{Listagem}: Apresenta uma lista de todos os candidatos registados na base de dados.
    \item \textbf{Detalhes}: Ao selecionar um candidato, o utilizador tem acesso a uma ficha detalhada que inclui:
    \begin{itemize}
        \item Biografia e dados pessoais (idade, profissão).
        \item Partido político e cargos desempenhados.
        \item Lista de propostas eleitorais principais.
    \end{itemize}
\end{itemize}

\section{Notícias em Tempo Real} \label{sec:func_noticias}
Para manter o utilizador atualizado sobre a atualidade política:

% [TODO: Inserir Screenshot Notícias]
\begin{figure}[H]
    \centering
    \includegraphics[width=0.4\textwidth]{images/screenshot_noticias.jpg}
    \caption{Feed de notícias políticas.}
    \label{fig:noticias}
\end{figure}

\begin{itemize}
    \item \textbf{Feed RSS}: Integração com o feed de política da RTP Notícias.
    \item \textbf{Pesquisa}: Barra de pesquisa que permite filtrar notícias por palavras-chave em tempo real.
    \item \textbf{Visualização}: Abertura da notícia completa num navegador interno ou externo.
\end{itemize}

\section{Sondagens e Estatísticas} \label{sec:func_sondagens}
A secção de sondagens permite visualizar as tendências de voto.

% [TODO: Inserir Screenshot Sondagem]
\begin{figure}[h!]
    \centering
    \includegraphics[width=0.4\textwidth]{images/screenshot_sondagens.jpg}
    \caption{Visualização gráfica dos resultados de uma sondagem.}
    \label{fig:sondagem}
\end{figure}

\begin{itemize}
    \item \textbf{Gráficos}: Visualização clara dos resultados através de gráficos de barras.
    \item \textbf{Ficha Técnica}: Informação sobre a amostra, margem de erro e empresa responsável pela sondagem.
\end{itemize}

\section{Datas Importantes} \label{sec:func_datas}
Um calendário eleitoral organizado por categorias que lista eventos cruciais. A interface foi reestruturada para utilizar um sistema de \textbf{abas (Tabs)}, o que permite ao utilizador filtrar facilmente entre:
\begin{itemize}
    \item \textbf{Datas}: Prazos oficiais, dias de reflexão e o dia das eleições.
    \item \textbf{Debates}: Calendário de debates televisivos entre candidatos.
    \item \textbf{Entrevistas}: Agendamento de entrevistas aos candidatos.
\end{itemize}
Esta organização facilita a consulta e garante que o eleitor não perde prazos ou eventos importantes.

\section{Notificações de Debates} \label{sec:func_notificacoes}
Para assegurar que os eleitores acompanham os debates televisivos, a aplicação dispõe de um sistema de notificações.

% [TODO: Inserir Screenshot Notificação]
\begin{figure}[H]
    \centering
    \includegraphics[width=0.4\textwidth]{images/screenshot_notificacao.png}
    \caption{Exemplo de notificação recebida antes e aquando de um debate.}
    \label{fig:notificacao_debate}
\end{figure}

\begin{itemize}
    \item \textbf{Alerta Antecipado}: O utilizador é notificado com antecedência sobre o início da transmissão.
    \item \textbf{Alerta ao Iniciar}: O utilizador é notificado imediatamente quando a transmissão começa.
    \item \textbf{Conteúdo}: A notificação indica os candidatos intervenientes e o canal televisivo.
\end{itemize}

\section{Bússola Política} \label{sec:func_compass}
A Bússola Política é uma ferramenta interativa que permite ao utilizador descobrir o seu posicionamento político através de um questionário. Esta funcionalidade foi inspirada no conhecido website \textit{The Political Compass} \cite{PoliticalCompass}, adaptando o conceito para o contexto da aplicação.

% [TODO: Inserir Screenshot Bússola]
\begin{figure}[H]
    \centering
    \includegraphics[width=0.4\textwidth]{images/screenshot_compass.jpg}
    \caption{Resultado do teste da Bússola Política.}
    \label{fig:compass}
\end{figure}

\begin{itemize}
    \item \textbf{Questionário}: Um conjunto de perguntas sobre temas económicos e sociais, onde o utilizador expressa o seu grau de concordância.
    \item \textbf{Visualização}: O resultado é apresentado num gráfico bidimensional (Eixo Económico vs. Eixo Social), e compara a posição do utilizador com a de vários candidatos e partidos.
    \item \textbf{Partilhar}: O utilizador pode partilhar o resultado do teste como imagem através de um botão.
\end{itemize}

\section{Petições Públicas} \label{sec:func_peticoes}
Para promover a participação cívica ativa, a aplicação inclui um sistema de petições públicas.

% [TODO: Inserir Screenshot Petições]
\begin{figure}[H]
    \centering
    \includegraphics[width=0.4\textwidth]{images/screenshot_peticoes.jpg}
    \caption{Lista de petições públicas criadas pelos utilizadores.}
    \label{fig:peticoes}
\end{figure}

\begin{itemize}
    \item \textbf{Criação}: Qualquer utilizador autenticado pode criar uma nova petição, na qual define um título, uma descrição e adiciona uma \textbf{imagem ilustrativa}.
    \item \textbf{Assinatura}: Os utilizadores podem apoiar causas ao assinar petições existentes.
    \item \textbf{Ordenação}: A lista de petições pode ser ordenada por popularidade (mais votadas) ou por data (mais recentes), o que facilita a descoberta de causas relevantes.
\end{itemize}

\section{Opções de Desenvolvedor (Administração)} \label{sec:func_dev_options}
Para facilitar a gestão de conteúdos e testes durante o desenvolvimento, foi implementada uma interface de administração acessível diretamente na aplicação.

% [TODO: Inserir Screenshot Dev Options]
\begin{figure}[H]
    \centering
    \includegraphics[width=0.4\textwidth]{images/screenshot_dev_ops.png}
    \caption{Interface de Opções de Desenvolvedor.}
    \label{fig:dev_options}
\end{figure}

Esta funcionalidade é acedida através de um botão flutuante no ecrã principal que apenas aparece para utilizadores com o atributo com a role: "admin", e permite:
\begin{itemize}
    \item \textbf{Adicionar Candidato}: Criação de novos perfis de candidatos, incluindo o upload de fotografia, biografia e dados partidários.
    \item \textbf{Adicionar Evento}: Agendamento de eventos (debates, arruadas) com suporte para geolocalização (latitude/longitude).
    \item \textbf{Criar Sondagem}: Lançamento de novas sondagens. Esta interface permite selecionar os candidatos participantes e \textbf{inserir manualmente} as percentagens de voto para cada um, permitindo simular diferentes cenários eleitorais para testes.
\end{itemize}

\section{Mapa Interativo} \label{sec:func_mapa}
Para fornecer uma componente visual geográfica, a aplicação integra um mapa interativo no ecrã principal.

% [TODO: Inserir Screenshot Mapa]
\begin{figure}[H]
    \centering
    \includegraphics[width=0.4\textwidth]{images/screenshot_mapa.jpg}
    \caption{Mapa interativo integrado no ecrã principal.}
    \label{fig:mapa}
\end{figure}

Esta funcionalidade permite visualizar a localização de eventos ou pontos de interesse relevantes para o contexto eleitoral, utilizando a tecnologia familiar do Google Maps.