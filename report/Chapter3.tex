\chapter{Funcionalidades} \label{chap:funcionalidades}

A aplicação VotoInformado oferece um conjunto de funcionalidades desenhadas para informar o eleitor.

\section{Autenticação e Perfil} \label{sec:func_auth}
A segurança e personalização são garantidas através de um sistema de autenticação robusto.

% [TODO: Inserir Screenshot Login]
\begin{figure}[H]
    \centering
    % \includegraphics[width=0.4\textwidth]{images/screenshot_login.png}
    \fbox{\begin{minipage}{0.4\textwidth}
        \centering
        \vspace{4cm}
        \textbf{[PLACEHOLDER: Screenshot Login]} \\
        (images/screenshot\_login.png)
        \vspace{4cm}
    \end{minipage}}
    \caption{Ecrã de Login da aplicação.}
    \label{fig:login}
\end{figure}

\begin{itemize}
    \item \textbf{Login e Registo}: Os utilizadores podem criar conta com email e password ou utilizar a sua conta Google para um acesso mais rápido.
    \item \textbf{Gestão de Sessão}: A aplicação mantém a sessão do utilizador ativa, permitindo acesso direto sem necessidade de login constante.
\end{itemize}

\section{Requisitos Funcionais} \label{sec:func_requisitos}
A tabela abaixo resume as principais funcionalidades implementadas.

% [TODO: Tabela de Requisitos]
\begin{table}[H]
    \centering
    \caption{Requisitos Funcionais do VotoInformado.}
    \begin{tabular}{|c|l|p{8cm}|}
        \hline
        \textbf{ID} & \textbf{Funcionalidade} & \textbf{Descrição} \\
        \hline
        RF01 & Autenticação & Permitir login e registo via Email/Password e Google. \\
        \hline
        RF02 & Listar Candidatos & Visualizar lista de candidatos com foto e partido. \\
        \hline
        RF03 & Detalhe Candidato & Ver biografia, propostas e cargos de um candidato. \\
        \hline
        RF04 & Feed Notícias & Ler notícias políticas atualizadas via RSS. \\
        \hline
        RF05 & Sondagens & Visualizar gráficos de intenção de voto. \\
        \hline
    \end{tabular}
    \label{tab:requisitos}
\end{table}

% [TODO: Inserir Diagrama de Casos de Uso]
\begin{figure}[H]
    \centering
    % \includegraphics[width=0.8\textwidth]{images/diagrama_casos_uso.png}
    \fbox{\begin{minipage}{0.8\textwidth}
        \centering
        \vspace{3cm}
        \textbf{[PLACEHOLDER: Diagrama de Casos de Uso]} \\
        (Atores: Eleitor, Admin; Casos: Ver Candidatos, Votar, Ler Notícias)
        \vspace{3cm}
    \end{minipage}}
    \caption{Diagrama de Casos de Uso do sistema.}
    \label{fig:use_cases}
\end{figure}

\section{Consulta de Candidatos} \label{sec:func_candidatos}
Esta é uma das funcionalidades centrais da aplicação.

% [TODO: Inserir Screenshot Lista Candidatos]
\begin{figure}[H]
    \centering
    % \includegraphics[width=0.4\textwidth]{images/screenshot_candidatos.png}
    \fbox{\begin{minipage}{0.4\textwidth}
        \centering
        \vspace{4cm}
        \textbf{[PLACEHOLDER: Screenshot Candidatos]} \\
        (images/screenshot\_candidatos.png)
        \vspace{4cm}
    \end{minipage}}
    \caption{Lista de candidatos apresentada ao utilizador.}
    \label{fig:candidatos}
\end{figure}

\begin{itemize}
    \item \textbf{Listagem}: Apresenta uma lista de todos os candidatos registados na base de dados.
    \item \textbf{Detalhes}: Ao selecionar um candidato, o utilizador tem acesso a uma ficha detalhada que inclui:
    \begin{itemize}
        \item Biografia e dados pessoais (idade, profissão).
        \item Partido político e cargos desempenhados.
        \item Lista de propostas eleitorais principais.
    \end{itemize}
\end{itemize}

\section{Notícias em Tempo Real} \label{sec:func_noticias}
Para manter o utilizador atualizado sobre a atualidade política:

% [TODO: Inserir Screenshot Notícias]
\begin{figure}[H]
    \centering
    % \includegraphics[width=0.4\textwidth]{images/screenshot_noticias.png}
    \fbox{\begin{minipage}{0.4\textwidth}
        \centering
        \vspace{4cm}
        \textbf{[PLACEHOLDER: Screenshot Notícias]} \\
        (images/screenshot\_noticias.png)
        \vspace{4cm}
    \end{minipage}}
    \caption{Feed de notícias políticas.}
    \label{fig:noticias}
\end{figure}

\begin{itemize}
    \item \textbf{Feed RSS}: Integração com o feed de política da RTP Notícias.
    \item \textbf{Pesquisa}: Barra de pesquisa que permite filtrar notícias por palavras-chave em tempo real.
    \item \textbf{Visualização}: Abertura da notícia completa num navegador interno ou externo.
\end{itemize}

\section{Sondagens e Estatísticas} \label{sec:func_sondagens}
A secção de sondagens permite visualizar as tendências de voto.

% [TODO: Inserir Screenshot Sondagem]
\begin{figure}[h!]
    \centering
    % \includegraphics[width=0.4\textwidth]{images/screenshot_sondagem.png}
    \fbox{\begin{minipage}{0.4\textwidth}
        \centering
        \vspace{4cm}
        \textbf{[PLACEHOLDER: Screenshot Sondagem]} \\
        (images/screenshot\_sondagem.png)
        \vspace{4cm}
    \end{minipage}}
    \caption{Visualização gráfica dos resultados de uma sondagem.}
    \label{fig:sondagem}
\end{figure}

\begin{itemize}
    \item \textbf{Gráficos}: Visualização clara dos resultados através de gráficos de barras.
    \item \textbf{Ficha Técnica}: Informação sobre a amostra, margem de erro e empresa responsável pela sondagem.
\end{itemize}

\section{Datas Importantes} \label{sec:func_datas}
Um calendário eleitoral que lista eventos cruciais como debates, dias de reflexão e o dia das eleições, garantindo que o eleitor não perde prazos importantes.