\chapter{Implementação} \label{chap:implementacao}

Este capítulo aborda os desafios técnicos e as soluções de implementação adotadas durante o desenvolvimento.

\section{Integração com Firestore} \label{sec:impl_firestore}
A migração de dados locais para a nuvem foi um passo fundamental. A implementação utiliza \texttt{CollectionReference} para aceder às coleções de ``candidatos'' e ``sondagens''. O uso de \textit{listeners} assíncronos permite que a interface seja atualizada automaticamente assim que os dados são recebidos, sem bloquear a \textit{Main Thread}.

% [TODO: Inserir Modelo de Dados]
\begin{figure}[H]
    \centering
    % \includegraphics[width=0.7\textwidth]{images/modelo_dados_firestore.png}
    \fbox{\begin{minipage}{0.7\textwidth}
        \centering
        \vspace{3cm}
        \textbf{[PLACEHOLDER: Modelo de Dados Firestore]} \\
        (Screenshot do Firebase Console ou Diagrama ER)
        \vspace{3cm}
    \end{minipage}}
    \caption{Estrutura das coleções no Cloud Firestore.}
    \label{fig:firestore_model}
\end{figure}

\begin{lstlisting}[language=Java, caption=Exemplo de leitura do Firestore]
db.collection("candidatos")
    .get()
    .addOnCompleteListener(task -> {
        if (task.isSuccessful()) {
            for (QueryDocumentSnapshot document : task.getResult()) {
                Candidato candidato = document.toObject(Candidato.class);
                listaCandidatos.add(candidato);
            }
            adapter.notifyDataSetChanged();
        }
    });
\end{lstlisting}

\section{Parsing de Notícias (RSS)} \label{sec:impl_rss}
Para obter as notícias, foi implementada a classe \texttt{NoticiasFetcher}. Esta classe realiza uma requisição HTTP ao feed RSS da RTP e faz o parsing do XML resultante utilizando \texttt{DocumentBuilder}.
Um desafio interessante foi a extração de imagens, que não vinham num campo explícito, mas sim embutidas na descrição HTML. Foi utilizada uma expressão regular (Regex) para extrair o atributo \texttt{src} das tags \texttt{<img>}.

% [TODO: Inserir Diagrama de Sequência]
\begin{figure}[H]
    \centering
    % \includegraphics[width=0.8\textwidth]{images/diagrama_sequencia_noticias.png}
    \fbox{\begin{minipage}{0.8\textwidth}
        \centering
        \vspace{3cm}
        \textbf{[PLACEHOLDER: Diagrama de Sequência]} \\
        (Fluxo: Fragment -> Fetcher -> HTTP Request -> XML Parsing -> Return List)
        \vspace{3cm}
    \end{minipage}}
    \caption{Diagrama de sequência do processo de obtenção de notícias.}
    \label{fig:seq_noticias}
\end{figure}

\section{Gestão de Imagens} \label{sec:impl_imagens}
O carregamento de imagens remotas em listas (\texttt{RecyclerView}) pode causar problemas de desempenho e consumo excessivo de memória. A biblioteca Picasso foi utilizada para mitigar estes problemas, gerindo automaticamente o download, cache e redimensionamento das imagens.

\section{Adapters e RecyclerViews} \label{sec:impl_adapters}
A exibição de listas eficientes foi conseguida através da implementação de \texttt{RecyclerView.Adapter} personalizados. O padrão \textit{ViewHolder} é utilizado para reciclar as vistas, garantindo uma rolagem suave mesmo com listas longas de candidatos ou notícias.
