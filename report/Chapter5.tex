\chapter{Resultados e Testes} \label{chap:resultados}

\section{Estado Atual do Projeto} \label{sec:res_estado}
A aplicação encontra-se num estado funcional estável, com todas as funcionalidades principais implementadas e operacionais. A navegação entre ecrãs é fluida e a integração com o Firebase responde com latência reduzida.

\section{Testes Realizados} \label{sec:res_testes}
Para garantir a qualidade do software, foram realizadas várias etapas de verificação:

\subsection{Testes Manuais}
Foram realizados testes exaustivos em emuladores e dispositivos físicos para validar:
\begin{itemize}
    \item O fluxo de registo e login.
    \item A correta visualização dos dados dos candidatos.
    \item A resiliência da aplicação a falhas de rede (tratamento de erros no carregamento de notícias).
    \item A adaptação da interface ao Modo Escuro.
\end{itemize}

% [TODO: Tabela de Casos de Teste]
\begin{table}[H]
    \centering
    \caption{Exemplo de casos de teste executados.}
    \begin{tabular}{|c|p{5cm}|p{5cm}|c|}
        \hline
        \textbf{ID} & \textbf{Ação} & \textbf{Resultado Esperado} & \textbf{Estado} \\
        \hline
        CT01 & Login com credenciais inválidas & Exibir mensagem de erro & Passou \\
        \hline
        CT02 & Carregar lista de notícias & Exibir lista com títulos e imagens & Passou \\
        \hline
        CT03 & Clicar em candidato & Abrir detalhe do candidato & Passou \\
        \hline
    \end{tabular}
    \label{tab:testes}
\end{table}

\subsection{Testes Unitários e Instrumentados}
O projeto inclui uma estrutura para testes unitários (JUnit) e testes de interface (Espresso), permitindo a verificação automática de componentes críticos e fluxos de utilizador.

\section{Análise de Desempenho} \label{sec:res_desempenho}
A utilização do \textit{Android Profiler} permitiu identificar e corrigir fugas de memória, especialmente no carregamento de imagens, o que confirma a eficácia da introdução da biblioteca Picasso.
