\chapter{Conclusão e Trabalho Futuro} \label{chap:conclusao}

\section{Conclusão} \label{sec:conc_conclusao}
O projeto VotoInformado atingiu os seus objetivos principais, o que resultou numa aplicação móvel capaz de informar os eleitores de forma clara e acessível. A evolução da arquitetura para uma \textbf{API intermédia em Node.js} com base de dados \textbf{MongoDB}, provou ser uma decisão acertada, o que garante maior segurança, centralização da lógica e escalabilidade da solução.

A aprendizagem adquirida durante o desenvolvimento, nomeadamente na gestão de dependências, chamadas assíncronas e design de interfaces Android, foi valiosa para a equipa.

\section{Trabalho Futuro} \label{sec:conc_futuro}
Apesar de funcional, a aplicação tem margem para evolução. Algumas sugestões para versões futuras incluem:
\begin{itemize}
    \item \textbf{Notificações Push}: Alertar os utilizadores para novas sondagens ou notícias urgentes.
    \item \textbf{Comparador de Candidatos}: Uma ferramenta para colocar lado a lado as propostas de dois candidatos.
    \item \textbf{Gamificação}: Introduzir quizzes sobre política para incentivar a aprendizagem de forma lúdica.
    \item \textbf{Suporte Offline}: Melhorar a cache de dados para permitir a consulta básica mesmo sem ligação à internet.
\end{itemize}