\begin{tabularx}{\linewidth}{l p{0.5cm} Y}
	\textbf{API} & & \textit{Application Programming Interface}. Uma interface que define interações entre múltiplas aplicações de software ou intermediários de hardware-software mistos. Permite que diferentes sistemas comuniquem entre si de forma padronizada. \\
	\textbf{Backend} & & A parte de um sistema ou aplicação de computador que não é diretamente acedida pelo utilizador, tipicamente responsável por armazenar e manipular dados, bem como pela lógica de negócio. \\
	\textbf{Frontend} & & A parte de um sistema ou aplicação de computador com a qual o utilizador interage diretamente. É sinónimo de interface de utilizador (UI). \\
	\textbf{Firebase} & & Uma plataforma desenvolvida pela Google para a criação de aplicações móveis e web. Fornece serviços como autenticação, bases de dados em tempo real, armazenamento e alojamento. \\
	\textbf{Git} & & Um sistema de controlo de versões distribuído para registar alterações no código-fonte durante o desenvolvimento de software. \\
	\textbf{JSON} & & \textit{JavaScript Object Notation}. Um formato leve de troca de dados, fácil de ler e escrever para humanos e fácil de interpretar e gerar para as máquinas. \\
	\textbf{MongoDB} & & Um programa de base de dados orientado a documentos e multiplataforma. Classificado como uma base de dados NoSQL, o MongoDB utiliza documentos do tipo JSON com esquemas opcionais. \\
	\textbf{Node.js} & & Um ambiente de execução de JavaScript de código aberto, multiplataforma e de backend que funciona no motor V8 e executa código JavaScript fora de um navegador web. \\
	\textbf{REST} & & \textit{Representational State Transfer}. Um estilo de arquitetura de software que define um conjunto de restrições a serem usadas para a criação de serviços Web. \\
	\textbf{Retrofit} & & Um cliente HTTP seguro para Android e Java, que facilita a comunicação com APIs REST. \\
	\textbf{SDK} & & \textit{Software Development Kit}. Uma coleção de ferramentas de desenvolvimento de software num único pacote instalável. \\
	\textbf{UI} & & \textit{User Interface} (Interface de Utilizador). O espaço onde ocorrem as interações entre humanos e máquinas. \\
	\textbf{UX} & & \textit{User Experience} (Experiência do Utilizador). As emoções e atitudes de uma pessoa sobre a utilização de um determinado produto, sistema ou serviço. \\
	\LaTeX & & Conjunto de macros para o processador de textos \TeX, utilizado amplamente para a produção de textos matemáticos e científicos devido à sua alta qualidade tipográfica.\\
\end{tabularx}