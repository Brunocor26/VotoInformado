\chapter{Introdução}\footnote{Este relatório foi produzido utilizando o template \LaTeX\ para teses e dissertações da Universidade da Beira Interior \cite{template-ubi-latex}, uma versão não oficial mantida pela comunidade.} \label{chap:intro}

\section{Enquadramento e Motivação} \label{sec:intro_motivacao}
O projeto \textbf{VotoInformado} surge no âmbito da unidade curricular de Programação de Dispositivos Móveis, com o intuito de combater a desinformação e o alheamento político. Numa era em que a informação é abundante mas nem sempre fidedigna, torna-se crucial fornecer aos cidadãos ferramentas que centralizem dados oficiais e notícias relevantes sobre o panorama político nacional.

A aplicação visa simplificar o acesso a informações sobre candidatos, partidos, sondagens e eventos eleitorais, de modo a promover uma participação cívica mais consciente e informada.

\section{Objetivos} \label{sec:intro_objetivos}
Os principais objetivos deste projeto são:
\begin{itemize}
    \item Desenvolver uma aplicação Android nativa, robusta e intuitiva.
    \item Centralizar informações sobre candidatos e partidos políticos.
    \item Disponibilizar sondagens eleitorais com visualização gráfica de dados.
    \item Fornecer um feed de notícias políticas atualizado em tempo real.
    \item Implementar um calendário de datas importantes para o processo eleitoral.
\end{itemize}

\section{Estrutura do Relatório} \label{sec:intro_estrutura}
Este relatório está organizado da seguinte forma:
\begin{itemize}
    \item O \textbf{Capítulo \ref{chap:arquitetura}} descreve a arquitetura do sistema e as tecnologias utilizadas.
    \item O \textbf{Capítulo \ref{chap:funcionalidades}} detalha as funcionalidades implementadas na aplicação.
    \item O \textbf{Capítulo \ref{chap:implementacao}} aborda aspetos técnicos da implementação e desafios encontrados.
    \item O \textbf{Capítulo \ref{chap:resultados}} apresenta os resultados obtidos e a validação da solução.
    \item O \textbf{Capítulo \ref{chap:conclusao}} apresenta as conclusões e sugestões para trabalho futuro.
\end{itemize}