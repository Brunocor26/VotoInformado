\documentclass[11pt,twoside]{template-ubi/Styling}

%\setmainfont{Georgia}

\usepackage{listings}
\lstset{
	language=C++,                % choose the language of the code
	basicstyle=\footnotesize,       % the size of the fonts that are used for the code
	numbers=left,                   % where to put the line-numbers
	numberstyle=\footnotesize,      % the size of the fonts that are used for the line-numbers
	stepnumber=1,                   % the step between two line-numbers. If it is 1 each line will be numbered
	numbersep=5pt,                  % how far the line-numbers are from the code
	backgroundcolor=\color{white},  % choose the background color. You must add \usepackage{color}
	showspaces=false,               % show spaces adding particular underscores
	showstringspaces=false,         % underline spaces within strings
	showtabs=false,                 % show tabs within strings adding particular underscores
	frame=single,           % adds a frame around the code
	tabsize=2,          % sets default tabsize to 2 spaces
	captionpos=b,           % sets the caption-position to bottom
	breaklines=true,        % sets automatic line breaking
	breakatwhitespace=false,    % sets if automatic breaks should only happen at whitespace
	escapeinside={\%*}{*)}          % if you want to add a comment within your code
}

% Comment this line if you are writing your thesis in English
% \portugues

%%Para gerar índice remissivo (que será inserido no documento com o comando \printindex)
\makeindex

\thesistitle{VotoInformado}
% Se existir um sub-título, descomente a linha abaixo e 
\thesissubtitle{Grupo 3}

%Nome da faculdade
\facultyname{Engenharia}

% Ciclo de estudos (1º, 2º ou 3º)
\studiescyclenumber{1\textsuperscript{o}}

\thesisauthors{Alexandre Santos, Bruno Correia, Daniel Carlos, Henrique Laia, Vasco Colaço}
\thesistype{Projeto Coletivo de Programação em Dispositivos Móveis}
\thesislocalanddate{Novembro de 2025}

\thesissupervisors{
	Orientador: Prof. Doutor Paulo Fazendeiro\\
	%Co-orientador: Prof. Doutor Name of your Co-Surpervisor\\
}

% Name of your course
\thesiscourse{Engenharia Informática}

% O comando seguinte insere o nome da tese no cabeçalho das páginas (comentar se não for pretendido)
% (Pode ser o título abreviado)
% \cabecalho{\thesistitlestr}

\begin{document}

% Abstract in Portuguese
\thesisresumo{
    A aplicação VotoInformado foi desenvolvida com o objetivo de fornecer aos eleitores informação fiável e acessível sobre candidatos, partidos, programas eleitorais e eventos políticos. Os objetivos centrais do projeto são centralizar a informação política, facilitar a comparação entre candidatos e aumentar o nível de literacia eleitoral da população. A solução consiste numa aplicação Android que inclui funcionalidades como listas de candidatos filtráveis, páginas de partidos , secção de notícias, sondagens, um questionário de posicionamento político e um sistema de criação de petições pelos utilizadores, suportado por Firebase.
}{VotoInformado, Aplicação Móvel, Android, Literacia Política, Eleições, Firebase}


\tableofcontents   % Índice
\listoffigures     % Lista de Figuras    (Opcional)
\listoftables      % Lista de Tabelas    (Opcional)
\listofalgorithms  % Lista de Algoritmos (Opcional)
\thesisacronyms{\begin{tabularx}{\linewidth}{l p{0.5cm} Y}
 	\textbf{API} & & Application Programming Interface (Interface de Programação de Aplicações) \\
 	\textbf{HTTP} & & Hypertext Transfer Protocol (Protocolo de Transferência de Hipertexto) \\
 	\textbf{JSON} & & JavaScript Object Notation \\
 	\textbf{JWT} & & JSON Web Token \\
 	\textbf{ODM} & & Object-Document Mapping (Mapeamento Objeto-Documento) \\
 	\textbf{REST} & & Representational State Transfer (Transferência de Estado Representacional) \\
 	\textbf{SDK} & & Software Development Kit (Kit de Desenvolvimento de Software) \\
 	\textbf{UI} & & User Interface (Interface do Utilizador) \\
 	\textbf{UX} & & User Experience (Experiência do Utilizador) \\
 	\textbf{XML} & & Extensible Markup Language (Linguagem de Marcação Extensível) \\
 	\textbf{UBI} & & Universidade da Beira Interior\\
\end{tabularx}}  % Lista de Acrónimos  (Opcional)

\mainmatter % inicia páginas em números arábicos a partir do 1

% Os capítulos são inseridos a partir daqui 
\chapter{Introdução}\footnote{Este relatório foi produzido utilizando o template \LaTeX\ para teses e dissertações da Universidade da Beira Interior \cite{template-ubi-latex}, uma versão não oficial mantida pela comunidade.} \label{chap:intro}

\section{Enquadramento e Motivação} \label{sec:intro_motivacao}
O projeto \textbf{VotoInformado} surge no âmbito da unidade curricular de Programação de Dispositivos Móveis, com o intuito de combater a desinformação e o alheamento político. Numa era em que a informação é abundante mas nem sempre fidedigna, torna-se crucial fornecer aos cidadãos ferramentas que centralizem dados oficiais e notícias relevantes sobre o panorama político nacional.

A aplicação visa simplificar o acesso a informações sobre candidatos, partidos, sondagens e eventos eleitorais, de modo a promover uma participação cívica mais consciente e informada.

\section{Objetivos} \label{sec:intro_objetivos}
Os principais objetivos deste projeto são:
\begin{itemize}
    \item Desenvolver uma aplicação Android nativa, robusta e intuitiva.
    \item Centralizar informações sobre candidatos e partidos políticos.
    \item Disponibilizar sondagens eleitorais com visualização gráfica de dados.
    \item Fornecer um feed de notícias políticas atualizado em tempo real.
    \item Implementar um calendário de datas importantes para o processo eleitoral.
\end{itemize}

\section{Estrutura do Relatório} \label{sec:intro_estrutura}
Este relatório está organizado da seguinte forma:
\begin{itemize}
    \item O \textbf{Capítulo \ref{chap:arquitetura}} descreve a arquitetura do sistema e as tecnologias utilizadas.
    \item O \textbf{Capítulo \ref{chap:funcionalidades}} detalha as funcionalidades implementadas na aplicação.
    \item O \textbf{Capítulo \ref{chap:implementacao}} aborda aspetos técnicos da implementação e desafios encontrados.
    \item O \textbf{Capítulo \ref{chap:resultados}} apresenta os resultados obtidos e a validação da solução.
    \item O \textbf{Capítulo \ref{chap:conclusao}} apresenta as conclusões e sugestões para trabalho futuro.
\end{itemize}
\chapter{Arquitetura e Tecnologias} \label{chap:arquitetura}

Este capítulo descreve a arquitetura de software adotada para o desenvolvimento da aplicação VotoInformado, bem como as tecnologias e bibliotecas utilizadas.

\section{Arquitetura do Sistema} \label{sec:arq_sistema}
A aplicação foi desenvolvida para o sistema operativo Android \cite{AndroidDev} e utiliza a linguagem Java. A estrutura do projeto segue os padrões recomendados pela Google e organiza o código em componentes lógicos para facilitar a manutenção e escalabilidade.

A arquitetura baseia-se na separação de responsabilidades:
\begin{itemize}
    \item \textbf{Activities e Fragments}: Responsáveis pela interface de utilizador (UI) e interação com o utilizador. Exemplos incluem \texttt{HomeActivity}, \texttt{NoticiasFragment} e \texttt{CandidatosFragment}.
    \item \textbf{Adapters}: Fazem a ponte entre os dados e os componentes de visualização de listas (\texttt{RecyclerView}).
    \item \textbf{Modelos (Classes)}: Representam as entidades de dados, como \texttt{Candidato}, \texttt{Noticia} e \texttt{Sondagem}.
    \item \textbf{Utils/Helpers}: Classes utilitárias para acesso a dados e operações comuns.
\end{itemize}

\section{Tecnologias Utilizadas} \label{sec:arq_tecnologias}

\subsection{Backend (API Customizada)}
O backend da aplicação, inicialmente projetado em Firebase, foi migrado para uma solução personalizada para responder a requisitos de maior complexidade técnica e controlo. A arquitetura atual utiliza:
\begin{itemize}
    \item \textbf{Node.js \& Express} \cite{NodeJS, ExpressJS}: Servidor de aplicação que expõe uma API RESTful para comunicação com a aplicação móvel.
    \item \textbf{MongoDB} \cite{MongoDB}: Base de dados NoSQL orientada a documentos, escolhida pela sua flexibilidade e escalabilidade.
    \item \textbf{Mongoose} \cite{Mongoose}: Biblioteca de modelação de objetos (ODM) para interação estruturada com o MongoDB.
\end{itemize}

\subsection{Firebase (Legado/Autenticação)}
Embora a persistência de dados tenha sido migrada, o Firebase continua a desempenhar um papel auxiliar:
\begin{itemize}
    \item \textbf{Firebase Authentication}: Mantido para gestão segura de identidades e sessões de utilizador.
    \item \textbf{Nota de Migração}: A transição do Firebase Firestore para a API personalizada foi motivada por uma sugestão pedagógica para aumentar a complexidade do projeto e demonstrar competências de desenvolvimento \textit{full-stack}.
\end{itemize}

\subsection{Bibliotecas Externas}
Para enriquecer a funcionalidade da aplicação, foram integradas diversas bibliotecas \textit{open-source}:

% [TODO: Tabela de Bibliotecas]
\begin{table}[H]
    \centering
    \caption{Bibliotecas externas utilizadas no projeto.}
    \begin{tabular}{|l|l|}
        \hline
        \textbf{Biblioteca} & \textbf{Propósito} \\
        \hline
        Picasso & Carregamento e colocação em cache de imagens assíncronos. \\
        \hline
        MPAndroidChart & Criação de gráficos para visualização de sondagens. \\
        \hline
        CircleImageView & Exibição de imagens de perfil circulares. \\
        \hline
        Retrofit/Gson & (Legado) Análise sintática de dados JSON. \\
        \hline
    \end{tabular}
    \label{tab:bibliotecas}
\end{table}

\begin{itemize}
    \item \textbf{Picasso} \cite{Picasso}: Biblioteca poderosa para o carregamento e cache de imagens em Android. Resolve problemas complexos de gestão de memória e reciclagem de vistas em listas.
    \item \textbf{MPAndroidChart} \cite{MPAndroidChart}: Utilizada para a criação de gráficos interativos e visualmente apelativos na secção de sondagens.
    \item \textbf{CircleImageView}: Permite a exibição de imagens de perfil com formato circular de forma simples e eficiente.
    \item \textbf{Retrofit/Gson} \cite{Retrofit} (Legado): Inicialmente utilizadas para análise sintática de JSON, foram substituídas pela integração direta com o Firestore, mas fizeram parte do processo de desenvolvimento.
\end{itemize}
\chapter{Funcionalidades} \label{chap:funcionalidades}

A aplicação VotoInformado oferece um conjunto de funcionalidades desenhadas para informar o eleitor.

\section{Autenticação e Perfil} \label{sec:func_auth}
A segurança e personalização são garantidas através de um sistema de autenticação robusto.

% [TODO: Inserir Screenshot Login]
\begin{figure}[H]
    \centering
    % \includegraphics[width=0.4\textwidth]{images/screenshot_login.png}
    \fbox{\begin{minipage}{0.4\textwidth}
        \centering
        \vspace{4cm}
        \textbf{[PLACEHOLDER: Screenshot Login]} \\
        (images/screenshot\_login.png)
        \vspace{4cm}
    \end{minipage}}
    \caption{Ecrã de Login da aplicação.}
    \label{fig:login}
\end{figure}

\begin{itemize}
    \item \textbf{Login e Registo}: Os utilizadores podem criar conta com email e password ou utilizar a sua conta Google para um acesso mais rápido.
    \item \textbf{Gestão de Sessão}: A aplicação mantém a sessão do utilizador ativa, permitindo acesso direto sem necessidade de login constante.
\end{itemize}

\section{Requisitos Funcionais} \label{sec:func_requisitos}
A tabela abaixo resume as principais funcionalidades implementadas.

% [TODO: Tabela de Requisitos]
\begin{table}[H]
    \centering
    \caption{Requisitos Funcionais do VotoInformado.}
    \begin{tabular}{|c|l|p{8cm}|}
        \hline
        \textbf{ID} & \textbf{Funcionalidade} & \textbf{Descrição} \\
        \hline
        RF01 & Autenticação & Permitir login e registo via Email/Password e Google. \\
        \hline
        RF02 & Listar Candidatos & Visualizar lista de candidatos com foto e partido. \\
        \hline
        RF03 & Detalhe Candidato & Ver biografia, propostas e cargos de um candidato. \\
        \hline
        RF04 & Feed Notícias & Ler notícias políticas atualizadas via RSS. \\
        \hline
        RF05 & Sondagens & Visualizar gráficos de intenção de voto. \\
        \hline
    \end{tabular}
    \label{tab:requisitos}
\end{table}

% [TODO: Inserir Diagrama de Casos de Uso]
\begin{figure}[H]
    \centering
    % \includegraphics[width=0.8\textwidth]{images/diagrama_casos_uso.png}
    \fbox{\begin{minipage}{0.8\textwidth}
        \centering
        \vspace{3cm}
        \textbf{[PLACEHOLDER: Diagrama de Casos de Uso]} \\
        (Atores: Eleitor, Admin; Casos: Ver Candidatos, Votar, Ler Notícias)
        \vspace{3cm}
    \end{minipage}}
    \caption{Diagrama de Casos de Uso do sistema.}
    \label{fig:use_cases}
\end{figure}

\section{Consulta de Candidatos} \label{sec:func_candidatos}
Esta é uma das funcionalidades centrais da aplicação.

% [TODO: Inserir Screenshot Lista Candidatos]
\begin{figure}[H]
    \centering
    % \includegraphics[width=0.4\textwidth]{images/screenshot_candidatos.png}
    \fbox{\begin{minipage}{0.4\textwidth}
        \centering
        \vspace{4cm}
        \textbf{[PLACEHOLDER: Screenshot Candidatos]} \\
        (images/screenshot\_candidatos.png)
        \vspace{4cm}
    \end{minipage}}
    \caption{Lista de candidatos apresentada ao utilizador.}
    \label{fig:candidatos}
\end{figure}

\begin{itemize}
    \item \textbf{Listagem}: Apresenta uma lista de todos os candidatos registados na base de dados.
    \item \textbf{Detalhes}: Ao selecionar um candidato, o utilizador tem acesso a uma ficha detalhada que inclui:
    \begin{itemize}
        \item Biografia e dados pessoais (idade, profissão).
        \item Partido político e cargos desempenhados.
        \item Lista de propostas eleitorais principais.
    \end{itemize}
\end{itemize}

\section{Notícias em Tempo Real} \label{sec:func_noticias}
Para manter o utilizador atualizado sobre a atualidade política:

% [TODO: Inserir Screenshot Notícias]
\begin{figure}[H]
    \centering
    % \includegraphics[width=0.4\textwidth]{images/screenshot_noticias.png}
    \fbox{\begin{minipage}{0.4\textwidth}
        \centering
        \vspace{4cm}
        \textbf{[PLACEHOLDER: Screenshot Notícias]} \\
        (images/screenshot\_noticias.png)
        \vspace{4cm}
    \end{minipage}}
    \caption{Feed de notícias políticas.}
    \label{fig:noticias}
\end{figure}

\begin{itemize}
    \item \textbf{Feed RSS}: Integração com o feed de política da RTP Notícias.
    \item \textbf{Pesquisa}: Barra de pesquisa que permite filtrar notícias por palavras-chave em tempo real.
    \item \textbf{Visualização}: Abertura da notícia completa num navegador interno ou externo.
\end{itemize}

\section{Sondagens e Estatísticas} \label{sec:func_sondagens}
A secção de sondagens permite visualizar as tendências de voto.

% [TODO: Inserir Screenshot Sondagem]
\begin{figure}[h!]
    \centering
    % \includegraphics[width=0.4\textwidth]{images/screenshot_sondagem.png}
    \fbox{\begin{minipage}{0.4\textwidth}
        \centering
        \vspace{4cm}
        \textbf{[PLACEHOLDER: Screenshot Sondagem]} \\
        (images/screenshot\_sondagem.png)
        \vspace{4cm}
    \end{minipage}}
    \caption{Visualização gráfica dos resultados de uma sondagem.}
    \label{fig:sondagem}
\end{figure}

\begin{itemize}
    \item \textbf{Gráficos}: Visualização clara dos resultados através de gráficos de barras.
    \item \textbf{Ficha Técnica}: Informação sobre a amostra, margem de erro e empresa responsável pela sondagem.
\end{itemize}

\section{Datas Importantes} \label{sec:func_datas}
Um calendário eleitoral que lista eventos cruciais como debates, dias de reflexão e o dia das eleições, garantindo que o eleitor não perde prazos importantes.

\section{Bússola Política} \label{sec:func_compass}
A Bússola Política é uma ferramenta interativa que permite ao utilizador descobrir o seu posicionamento político através de um questionário.

% [TODO: Inserir Screenshot Bússola]
\begin{figure}[H]
    \centering
    % \includegraphics[width=0.4\textwidth]{images/screenshot_compass.png}
    \fbox{\begin{minipage}{0.4\textwidth}
        \centering
        \vspace{4cm}
        \textbf{[PLACEHOLDER: Screenshot Bússola]} \\
        (images/screenshot\_compass.png)
        \vspace{4cm}
    \end{minipage}}
    \caption{Resultado do teste da Bússola Política.}
    \label{fig:compass}
\end{figure}

\begin{itemize}
    \item \textbf{Questionário}: Um conjunto de perguntas sobre temas económicos e sociais, onde o utilizador expressa o seu grau de concordância.
    \item \textbf{Visualização}: O resultado é apresentado num gráfico bidimensional (Eixo Económico vs. Eixo Social), comparando a posição do utilizador com a de vários candidatos e partidos.
\end{itemize}

\section{Petições Públicas} \label{sec:func_peticoes}
Para promover a participação cívica ativa, a aplicação inclui um sistema de petições públicas.

% [TODO: Inserir Screenshot Petições]
\begin{figure}[H]
    \centering
    % \includegraphics[width=0.4\textwidth]{images/screenshot_peticoes.png}
    \fbox{\begin{minipage}{0.4\textwidth}
        \centering
        \vspace{4cm}
        \textbf{[PLACEHOLDER: Screenshot Petições]} \\
        (images/screenshot\_peticoes.png)
        \vspace{4cm}
    \end{minipage}}
    \caption{Lista de petições públicas criadas pelos utilizadores.}
    \label{fig:peticoes}
\end{figure}

\begin{itemize}
    \item \textbf{Criação}: Qualquer utilizador autenticado pode criar uma nova petição, definindo um título, uma descrição e adicionando uma \textbf{imagem ilustrativa}.
    \item \textbf{Assinatura}: Os utilizadores podem apoiar causas assinando petições existentes.
    \item \textbf{Ordenação}: A lista de petições pode ser ordenada por popularidade (mais votadas) ou por data (mais recentes), facilitando a descoberta de causas relevantes.
\end{itemize}
\chapter{Implementação} \label{chap:implementacao}

Este capítulo aborda os desafios técnicos e as soluções de implementação adotadas durante o desenvolvimento.

\section{Integração com Firestore} \label{sec:impl_firestore}
A migração de dados locais para a nuvem foi um passo fundamental. A implementação utiliza \texttt{CollectionReference} para aceder às coleções de ``candidatos'' e ``sondagens''. O uso de \textit{listeners} assíncronos permite que a interface seja atualizada automaticamente assim que os dados são recebidos, sem bloquear a \textit{Main Thread}.

% [TODO: Inserir Modelo de Dados]
\begin{figure}[H]
    \centering
    % \includegraphics[width=0.7\textwidth]{images/modelo_dados_firestore.png}
    \fbox{\begin{minipage}{0.7\textwidth}
        \centering
        \vspace{3cm}
        \textbf{[PLACEHOLDER: Modelo de Dados Firestore]} \\
        (Screenshot do Firebase Console ou Diagrama ER)
        \vspace{3cm}
    \end{minipage}}
    \caption{Estrutura das coleções no Cloud Firestore.}
    \label{fig:firestore_model}
\end{figure}

\begin{lstlisting}[language=Java, caption=Exemplo de leitura do Firestore]
db.collection("candidatos")
    .get()
    .addOnCompleteListener(task -> {
        if (task.isSuccessful()) {
            for (QueryDocumentSnapshot document : task.getResult()) {
                Candidato candidato = document.toObject(Candidato.class);
                listaCandidatos.add(candidato);
            }
            adapter.notifyDataSetChanged();
        }
    });
\end{lstlisting}

\section{Parsing de Notícias (RSS)} \label{sec:impl_rss}
Para obter as notícias, foi implementada a classe \texttt{NoticiasFetcher}. Esta classe realiza uma requisição HTTP ao feed RSS da RTP e faz o parsing do XML resultante utilizando \texttt{DocumentBuilder}.
Um desafio interessante foi a extração de imagens, que não vinham num campo explícito, mas sim embutidas na descrição HTML. Foi utilizada uma expressão regular (Regex) para extrair o atributo \texttt{src} das tags \texttt{<img>}.

% [TODO: Inserir Diagrama de Sequência]
\begin{figure}[H]
    \centering
    % \includegraphics[width=0.8\textwidth]{images/diagrama_sequencia_noticias.png}
    \fbox{\begin{minipage}{0.8\textwidth}
        \centering
        \vspace{3cm}
        \textbf{[PLACEHOLDER: Diagrama de Sequência]} \\
        (Fluxo: Fragment -> Fetcher -> HTTP Request -> XML Parsing -> Return List)
        \vspace{3cm}
    \end{minipage}}
    \caption{Diagrama de sequência do processo de obtenção de notícias.}
    \label{fig:seq_noticias}
\end{figure}

\section{Gestão de Imagens} \label{sec:impl_imagens}
O carregamento de imagens remotas em listas (\texttt{RecyclerView}) pode causar problemas de desempenho e consumo excessivo de memória. A biblioteca Picasso foi utilizada para mitigar estes problemas, gerindo automaticamente o download, cache e redimensionamento das imagens.

\section{Adapters e RecyclerViews} \label{sec:impl_adapters}
A exibição de listas eficientes foi conseguida através da implementação de \texttt{RecyclerView.Adapter} personalizados. O padrão \textit{ViewHolder} é utilizado para reciclar as vistas, garantindo uma rolagem suave mesmo com listas longas de candidatos ou notícias.

\chapter{Resultados e Testes} \label{chap:resultados}

\section{Estado Atual do Projeto} \label{sec:res_estado}
A aplicação encontra-se num estado funcional estável, com todas as funcionalidades principais implementadas e operacionais. A navegação entre ecrãs é fluida e a integração com o Firebase responde com latência reduzida.

\section{Testes Realizados} \label{sec:res_testes}
Para garantir a qualidade do software, foram realizadas várias etapas de verificação:

\subsection{Testes Manuais}
Foram realizados testes exaustivos em emuladores e dispositivos físicos para validar:
\begin{itemize}
    \item O fluxo de registo e login.
    \item A correta visualização dos dados dos candidatos.
    \item A resiliência da aplicação a falhas de rede (tratamento de erros no carregamento de notícias).
    \item A adaptação da interface ao Modo Escuro.
\end{itemize}

% [TODO: Tabela de Casos de Teste]
\begin{table}[H]
    \centering
    \caption{Exemplo de casos de teste executados.}
    \begin{tabular}{|c|p{5cm}|p{5cm}|c|}
        \hline
        \textbf{ID} & \textbf{Ação} & \textbf{Resultado Esperado} & \textbf{Estado} \\
        \hline
        CT01 & Login com credenciais inválidas & Exibir mensagem de erro & Passou \\
        \hline
        CT02 & Carregar lista de notícias & Exibir lista com títulos e imagens & Passou \\
        \hline
        CT03 & Clicar em candidato & Abrir detalhe do candidato & Passou \\
        \hline
    \end{tabular}
    \label{tab:testes}
\end{table}

\subsection{Testes Unitários e Instrumentados}
O projeto inclui uma estrutura para testes unitários (JUnit) e testes de interface (Espresso), permitindo a verificação automática de componentes críticos e fluxos de utilizador.

\section{Análise de Desempenho} \label{sec:res_desempenho}
A utilização do \textit{Android Profiler} permitiu identificar e corrigir fugas de memória, especialmente no carregamento de imagens, o que confirma a eficácia da introdução da biblioteca Picasso.

\chapter{Conclusão e Trabalho Futuro} \label{chap:conclusao}

\section{Conclusão} \label{sec:conc_conclusao}
O projeto VotoInformado atingiu os seus objetivos principais, o que resultou numa aplicação móvel capaz de informar os eleitores de forma clara e acessível. A evolução da arquitetura para uma \textbf{API intermédia em Node.js} com base de dados \textbf{MongoDB}, provou ser uma decisão acertada, o que garante maior segurança, centralização da lógica e escalabilidade da solução.

A aprendizagem adquirida durante o desenvolvimento, nomeadamente na gestão de dependências, chamadas assíncronas e design de interfaces Android, foi valiosa para a equipa.

\section{Trabalho Futuro} \label{sec:conc_futuro}
Apesar de funcional, a aplicação tem margem para evolução. Algumas sugestões para versões futuras incluem:
\begin{itemize}
    \item \textbf{Notificações Push}: Alertar os utilizadores para novas sondagens ou notícias urgentes.
    \item \textbf{Comparador de Candidatos}: Uma ferramenta para colocar lado a lado as propostas de dois candidatos.
    \item \textbf{Gamificação}: Introduzir quizzes sobre política para incentivar a aprendizagem de forma lúdica.
    \item \textbf{Suporte Offline}: Melhorar a cache de dados para permitir a consulta básica mesmo sem ligação à internet.
\end{itemize}

% Fim da inserção dos capítulos

% Bibliografia
% Primeiro parâmetro é o estilo e o segundo o arquivo bib
%\thesisbibliography{template-ubi/BiblioStyle}{bibliography}

% Exemplo de uso de outro estilo bibliográfico. Define ser definido apenas um estilo 
\thesisbibliography{template-ubi/IEEEtran}{bibliography}

\appendix{\chapter{Anexos} \label{chap:appendix}

Esta secção contém excertos de código relevantes que detalham a implementação do sistema VotoInformado.

\section{Código Fonte Relevante}

\subsection*{Cliente da API - ApiClient.java}
\begin{lstlisting}[language=Java, caption=Cliente da API para comunicação com o servidor.]
package pt.ubi.pdm.votoinformado.api;

import android.util.Log;
import java.io.IOException;
import java.net.HttpURLConnection;
import java.net.URL;
import java.util.concurrent.CountDownLatch;
import retrofit2.Retrofit;
import retrofit2.converter.gson.GsonConverterFactory;

public class ApiClient {
    private static final String RENDER_URL = "https://api-votoinformado.onrender.com/";
    private static final String LOCAL_URL = "http://10.0.2.2:3000/"; // Localhost for Android Emulator
    private static final String LOCAL_IP_URL = "http://10.250.134.7:3000/"; // Local IP for Physical Device

    private static volatile ApiClient instance;
    private static final Object lock = new Object();
    private final Retrofit retrofit;
    private static String baseUrl;

    private ApiClient(Retrofit retrofit) {
        this.retrofit = retrofit;
    }

    private static boolean isUrlAvailable(String urlString) {
        try {
            URL url = new URL(urlString);
            HttpURLConnection connection = (HttpURLConnection) url.openConnection();
            connection.setRequestMethod("HEAD");
            connection.setConnectTimeout(15000); // 15 seconds timeout for Render cold start
            connection.setReadTimeout(15000);
            connection.connect();
            int responseCode = connection.getResponseCode();
            return (responseCode >= 200 && responseCode < 300);
        } catch (IOException e) {
            return false;
        }
    }

    public static ApiClient getInstance() {
        if (instance == null) {
            synchronized (lock) {
                if (instance == null) {
                    CountDownLatch latch = new CountDownLatch(1);
                    new Thread(() -> {
                        if (isUrlAvailable(RENDER_URL)) {
                            baseUrl = RENDER_URL;
                        } else if (isUrlAvailable(LOCAL_IP_URL)) {
                            baseUrl = LOCAL_IP_URL;
                        } else {
                            baseUrl = LOCAL_URL;
                        }
                        Log.d("ApiClient", "Using base URL: " + baseUrl);

                        Retrofit retrofit = new Retrofit.Builder()
                                .baseUrl(baseUrl)
                                .addConverterFactory(GsonConverterFactory.create())
                                .build();
                        
                        instance = new ApiClient(retrofit);
                        latch.countDown();
                    }).start();

                    try {
                        latch.await(); // Block until initialization is complete
                    } catch (InterruptedException e) {
                        Thread.currentThread().interrupt();
                        // Handle error, maybe fallback to a default
                        if (instance == null) {
                           Log.e("ApiClient", "Initialization interrupted, falling back to local URL");
                           baseUrl = LOCAL_URL;
                           Retrofit retrofit = new Retrofit.Builder()
                                   .baseUrl(baseUrl)
                                   .addConverterFactory(GsonConverterFactory.create())
                                   .build();
                           instance = new ApiClient(retrofit);
                        }
                    }
                }
            }
        }
        return instance;
    }

    public ApiService getApiService() {
        return retrofit.create(ApiService.class);
    }

    public static String getBaseUrl() {
        if (instance == null) {
            getInstance(); // ensure initialization
        }
        return baseUrl;
    }
}
\end{lstlisting}

\subsection*{Atividade Principal - HomeActivity.java}
\begin{lstlisting}[language=Java, caption=Atividade principal que gere a navegação e o fragmento inicial.]
package pt.ubi.pdm.votoinformado.activities;

import android.Manifest;
import android.content.Intent;
import android.content.pm.PackageManager;
import android.os.Build;
import android.os.Bundle;
import android.view.MenuItem;

import androidx.activity.result.ActivityResultLauncher;
import androidx.activity.result.contract.ActivityResultContracts;
import androidx.annotation.NonNull;
import androidx.appcompat.app.AppCompatActivity;
import androidx.core.content.ContextCompat;
import androidx.fragment.app.Fragment;
import androidx.work.ExistingPeriodicWorkPolicy;
import androidx.work.PeriodicWorkRequest;
import androidx.work.WorkManager;

import com.google.android.material.bottomnavigation.BottomNavigationView;

import java.util.concurrent.TimeUnit;

import pt.ubi.pdm.votoinformado.R;
import pt.ubi.pdm.votoinformado.activities.notificacoes.SyncDatesWorker;
import pt.ubi.pdm.votoinformado.fragments.CandidatosFragment;
import pt.ubi.pdm.votoinformado.fragments.ChooseEventTypeFragment;
import pt.ubi.pdm.votoinformado.fragments.HomeFragment;
import pt.ubi.pdm.votoinformado.fragments.NoticiasFragment;
import pt.ubi.pdm.votoinformado.fragments.PeticoesFragment;
import pt.ubi.pdm.votoinformado.fragments.SondagensFragment;

public class HomeActivity extends AppCompatActivity implements BottomNavigationView.OnNavigationItemSelectedListener {

    // Launcher para pedir a permissão de notificação
    private final ActivityResultLauncher<String> requestPermissionLauncher = registerForActivityResult(new ActivityResultContracts.RequestPermission(), isGranted -> {
    });

    @Override
    protected void onCreate(Bundle savedInstanceState) {
        super.onCreate(savedInstanceState);

        // Check if user is logged in via shared preferences
        android.content.SharedPreferences prefs = getSharedPreferences("user_session", MODE_PRIVATE);
        String token = prefs.getString("auth_token", null);

        if (token == null || token.isEmpty()) {
            startActivity(new Intent(this, LoginActivity.class));
            finish();
            return;
        }

        setContentView(R.layout.activity_home);

        BottomNavigationView bottomNav = findViewById(R.id.bottom_navigation);
        bottomNav.setOnNavigationItemSelectedListener(this);

        // Load the default fragment
        if (savedInstanceState == null) {
            loadFragment(new HomeFragment());
        }

        // Pede permissão de notificações (para Android 13+)
        askNotificationPermission();
        
        //funcao que vai tratar das notificacoes mesmo sem a necessidade de ter a aplicacao aberta
        scheduleDateSync();
    }

    private void askNotificationPermission() {
        // Apenas necessário para API 33+ (Android 13)
        if (Build.VERSION.SDK_INT >= Build.VERSION_CODES.TIRAMISU) {
            if (ContextCompat.checkSelfPermission(this, Manifest.permission.POST_NOTIFICATIONS) != PackageManager.PERMISSION_GRANTED) {
                // Pede diretamente a permissão
                requestPermissionLauncher.launch(Manifest.permission.POST_NOTIFICATIONS);
            }
        }
    }

    private void scheduleDateSync() {
        //criamos um pedido de trabalho periódico
        PeriodicWorkRequest syncDatesRequest =
                new PeriodicWorkRequest.Builder(SyncDatesWorker.class, 24, TimeUnit.HOURS)
                        .build();               //executar a logica do SyncDatesWorker.class a cada 24h

        //entregamos o pedido de trabalho ao WorkManager
        WorkManager.getInstance(this).enqueueUniquePeriodicWork(
                "syncDatesWork",
                ExistingPeriodicWorkPolicy.KEEP,
                syncDatesRequest);
    }

    private boolean loadFragment(Fragment fragment) {
        if (fragment != null) {
            getSupportFragmentManager().beginTransaction()
                    .replace(R.id.fragment_container, fragment)
                    .commit();
            return true;
        }
        return false;
    }

    @Override
    public boolean onNavigationItemSelected(@NonNull MenuItem item) {
        Fragment fragment = null;

        int itemId = item.getItemId();
        if (itemId == R.id.nav_home) {
            fragment = new HomeFragment();
        } else if (itemId == R.id.nav_candidatos) {
            fragment = new CandidatosFragment();
        } else if (itemId == R.id.nav_eventos) {
            fragment = new pt.ubi.pdm.votoinformado.fragments.ImportantDatesHostFragment();
        } else if (itemId == R.id.nav_sondagens) {
            fragment = new SondagensFragment();
        } else if (itemId == R.id.nav_noticias) {
            fragment = new NoticiasFragment();
        } else if (itemId == R.id.nav_peticoes) {
            fragment = new PeticoesFragment();
        }

        return loadFragment(fragment);
    }
}
\end{lstlisting}

\subsection*{Modelo de Dados - Candidato.java}
\begin{lstlisting}[language=Java, caption=Classe que representa o modelo de dados de um Candidato.]
package pt.ubi.pdm.votoinformado.classes;

import android.annotation.SuppressLint;
import android.content.Context;
import android.util.Log;

    // @Exclude removed

import java.io.Serializable;

import pt.ubi.pdm.votoinformado.R;

import com.google.gson.annotations.SerializedName;

public class Candidato implements Serializable {

    @SerializedName("_id")
    private String id;

    @SerializedName("id")
    private String stringId;

    private String nome;
    private String partido;
    // private String fotoNome;
    private String profissao;
    private String cargosPrincipais;
    private String biografiaCurta;
    private String siteOficial;

    public Candidato() {
        // Construtor vazio necessário para o Firestore
    }

    // Getters and Setters with PropertyName annotations

    public String getId() { return id; }
    public void setId(String id) { this.id = id; }

    public String getStringId() { return stringId; }
    public void setStringId(String stringId) { this.stringId = stringId; }

    public String getNome() { return nome; }
    public void setNome(String nome) { this.nome = nome; }

    public String getPartido() { return partido; }
    public void setPartido(String partido) { this.partido = partido; }

    // fotoNome removed


    public String getProfissao() { return profissao; }
    public void setProfissao(String profissao) { this.profissao = profissao; }

    public String getCargosPrincipais() { return cargosPrincipais; }
    public void setCargosPrincipais(String cargosPrincipais) { this.cargosPrincipais = cargosPrincipais; }

    public String getBiografiaCurta() { return biografiaCurta; }
    public void setBiografiaCurta(String biografiaCurta) { this.biografiaCurta = biografiaCurta; }

    public String getSiteOficial() { return siteOficial; }
    public void setSiteOficial(String siteOficial) { this.siteOficial = siteOficial; }

    @SerializedName("photoUrl")
    private String photoUrl;

    public String getPhotoUrl() {
        return photoUrl;
    }

    public void setPhotoUrl(String photoUrl) {
        this.photoUrl = photoUrl;
    }
}
\end{lstlisting}} % Anexos (Opcional)
\thesisglossary{\begin{tabularx}{\linewidth}{l p{0.5cm} Y}
	\textbf{API} & & \textit{Application Programming Interface}. Uma interface que define interações entre múltiplas aplicações de software ou intermediários de hardware-software mistos. Permite que diferentes sistemas comuniquem entre si de forma padronizada. \\
	\textbf{Backend} & & A parte de um sistema ou aplicação de computador que não é diretamente acedida pelo utilizador, tipicamente responsável por armazenar e manipular dados, bem como pela lógica de negócio. \\
	\textbf{Frontend} & & A parte de um sistema ou aplicação de computador com a qual o utilizador interage diretamente. É sinónimo de interface de utilizador (UI). \\
	\textbf{Firebase} & & Uma plataforma desenvolvida pela Google para a criação de aplicações móveis e web. Fornece serviços como autenticação, bases de dados em tempo real, armazenamento e alojamento. \\
	\textbf{Git} & & Um sistema de controlo de versões distribuído para registar alterações no código-fonte durante o desenvolvimento de software. \\
	\textbf{JSON} & & \textit{JavaScript Object Notation}. Um formato leve de troca de dados, fácil de ler e escrever para humanos e fácil de interpretar e gerar para as máquinas. \\
	\textbf{MongoDB} & & Um programa de base de dados orientado a documentos e multiplataforma. Classificado como uma base de dados NoSQL, o MongoDB utiliza documentos do tipo JSON com esquemas opcionais. \\
	\textbf{Node.js} & & Um ambiente de execução de JavaScript de código aberto, multiplataforma e de backend que funciona no motor V8 e executa código JavaScript fora de um navegador web. \\
	\textbf{REST} & & \textit{Representational State Transfer}. Um estilo de arquitetura de software que define um conjunto de restrições a serem usadas para a criação de serviços Web. \\
	\textbf{Retrofit} & & Um cliente HTTP seguro para Android e Java, que facilita a comunicação com APIs REST. \\
	\textbf{SDK} & & \textit{Software Development Kit}. Uma coleção de ferramentas de desenvolvimento de software num único pacote instalável. \\
	\textbf{UI} & & \textit{User Interface} (Interface de Utilizador). O espaço onde ocorrem as interações entre humanos e máquinas. \\
	\textbf{UX} & & \textit{User Experience} (Experiência do Utilizador). As emoções e atitudes de uma pessoa sobre a utilização de um determinado produto, sistema ou serviço. \\
	\LaTeX & & Conjunto de macros para o processador de textos \TeX, utilizado amplamente para a produção de textos matemáticos e científicos devido à sua alta qualidade tipográfica.\\
\end{tabularx}}  % Glossário (Opcional)

\printindex % Inserir índice remissivo (Opcional)

\end{document}
